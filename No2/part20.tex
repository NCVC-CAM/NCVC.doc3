%!TEX root = ../NCVC3.tex

\mysection{加工条件の設定}

\vspace*{1zh}
 旋盤用の条件ファイルで設定します。拡張子はncjとなります。
以下に旋盤加工特有の設定ダイアログを列挙します。
フライス加工と違って送り速度に単位がない箇所があります。
G98毎分送りかG99毎回転送りによって変わりますので適宜読み替えてください。
送り速度に小数点が付く場合は[表記]タブの[Fパラメータ表記]を[小数点]にしておくと良いでしょう。

%\begin{minipage}{0.5\textwidth}
%\begin{figure}[H]
%\centering
%\includegraphics[scale=0.7]{No2/fig/ncj1.png}
%\caption{端面処理設定}
%\label{fig:ncj1.png}
%\end{figure}
%\end{minipage}
%\begin{minipage}{0.5\textwidth}
%\begin{figure}[H]
%\centering
%\includegraphics[scale=0.7]{No2/fig/ncj2.png}
%\caption{下穴加工設定}
%\label{fig:ncj2.png}
%\end{figure}
%\end{minipage}

%\begin{minipage}{0.5\textwidth}
%\begin{figure}[H]
%\centering
%\includegraphics[scale=0.7]{No2/fig/ncj3.png}
%\caption{内径加工設定}
%\label{fig:ncj3.png}
%\end{figure}
%\end{minipage}
%\begin{minipage}{0.5\textwidth}
%\begin{figure}[H]
%\centering
%\includegraphics[scale=0.7]{No2/fig/ncj4.png}
%\caption{外径加工設定}
%\label{fig:ncj4.png}
%\end{figure}
%\end{minipage}

%\begin{figure}[H]
%\centering
%\includegraphics[scale=0.7]{No2/fig/ncj5.png}
%\caption{突切(溝)加工設定}
%\label{fig:ncj5.png}
%\end{figure}

\begin{itemize}
\item 端面\\
 端面処理を行いたい場合は[端面処理を行う]にチェックを入れてください。
[カスタムコード]には工具交換などのコードが挿入できます。``\,\yen{}n\,''で改行できるので複数行のブロックも挿入可能です。

\vspace*{1zh}
\item 下穴\\
 ドリルによる穴加工を行いたい場合は[ドリル]に使用するドリル径を入力してください。
空白の場合は下穴加工データを生成しません。複数のドリルを使用する場合はコンマで区切ります。
回転数と送り速度も同様にコンマで区切ります。中心にしか切削データを生成できません。
複合機のようなY軸移動はできません。汎用旋盤における芯押し台のイメージです。\\
 工具主軸回転などの特殊コード挿入には[カスタムコード]を利用してください。
端面処理と同様に``\,\yen{}n\,''で改行できます。\\
 被削材が加工前すでに中空の場合は[既存下穴サイズ]に入力してください。
生成データ中に(LatheHole=〇〇)のコメントが埋め込まれOpenGLソリッド表示の描画に反映されます。
[固定サイクルで生成]にチェックが入ると、G83固定サイクルモードで加工データが生成されます。\\
 ドリル切削か既存下穴サイズが無いと次の内径切削で図7のエラーが表示されます。

%\begin{figure}[H]
%\centering
%\includegraphics[scale=0.7]{No2/fig/error.png}
%\caption{内径生成時のエラー}
%\label{fig:error.png}
%\end{figure}

\vspace*{1zh}
\item 内径と外径\\
 荒取りと形状の仕上げ工程で切削データが生成されますが、荒取り工程では図8のように計算されます。
これの逆形状、つまり外径では右(端面に近い方)に太く左(主軸に近い方)に細い、内径では右に細く左に太い形状は、
荒取り工程の座標計算ができない場合があるのでご注意ください。
この場合は被削材を反対に取り付けるなど、切削工程の見直しが必要です。

%\begin{figure}[H]
%\centering
%\includegraphics{No2/fig/inout.pdf}
%\caption{内外径の荒取り}
%\label{fig:inout.pdf}
%\end{figure}

\vspace*{1zh}
\item 突切\\
 作図した線の長さ(Z軸方向)が刃幅設定よりも長い場合と短い場合で生成される切削コードが違います。
図9のように長い場合はその線の長さ分だけZ軸方向の切削コードが生成されますが、短い場合はX軸方向を往復する切削コードが生成されます。
突っ切りバイトで前者の切削は仕上げ等で使うかもしれませんが、切り込み量などにご注意ください。通常後者になると思われます。

%\begin{minipage}{0.5\textwidth}
%\begin{figure}[H]
%\centering
%\includegraphics{No2/fig/groove1.pdf}
%\caption{作図線と切削コードの関係}
%\label{fig:ngroove1.pdf}
%\end{figure}
%\end{minipage}
%\begin{minipage}{0.5\textwidth}
%\begin{figure}[H]
%\centering
%\includegraphics{No2/fig/groove2.pdf}
%\caption{工具基準点}
%\label{fig:groove2.pdf}
%\end{figure}
%\end{minipage}

 工具基準点は図10のようになっています。生成される座標が作図した線の左・中央・右になります。
\end{itemize}

 サンプルのカスタムヘッダー・フッターをリスト1に示します。
章の冒頭で述べたように旋盤に必要なG99毎回点送りやG96周速一定制御指示などを追加してください。
さらにカスタムヘッダー・フッターで使用可能な旋盤生成に関する置換キーワードを表1に示します。

\begin{minipage}[t]{0.75\textwidth}
\begin{lstlisting}[caption=Header.txt,numbers=none,label=lst:header.txt]
%
({MakeDate} {MakeTime})
({MakeUser} MADE {MakeNCD} FROM {MakeDXF} AND {MakeCondition})
{G90orG91}G54{G92_Initial}
M8
{Spindle}M3
\end{lstlisting}
\end{minipage}
\begin{minipage}[t]{0.25\textwidth}
\begin{lstlisting}[caption=Footer.txt,numbers=none,label=lst:footer.txt]
M9
M5
{G0XY_Initial}
M30
%
\end{lstlisting}
\end{minipage}

\begin{table}[H]
\centering
\begin{tabular}{|p{3cm}|p{10cm}|}
\hline
MakeDate & \smash{\raisebox{-1zh}{生成した日付と時間に置換}} \\
MakeTime & \\ \hline
MakeUser & 生成したユーザで置換.但し漢字ユーザ名は[???]\\ \hline
MakeNCD  & 生成したNCファイル名 \\ \hline
MakeDXF  & 生成元のCADデータのファイル名 \\ \hline
MakeCondition & 生成時に参照した加工条件ファイル名 \\ \hline
G90orG91 & アブソリュートかインクリメントか \\ \hline
G92\_Initial & G92X\_Y\_Z\_に置換.座標値は基本タブから取得 \\ \hline
G92X & \\
G92Y & 切削原点(G92)のそれぞれの値に置換 \\
G92Z & \\ \hline
G0XY\_Initial & G00X\_Y\_に置換.座標値は基本タブから取得 \\ \hline
Spindle & 主軸回転命令 S\_ に置換 \\ \hline
\end{tabular}
\end{table}
